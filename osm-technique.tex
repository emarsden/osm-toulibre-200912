%% osm-technique.tex



\frame[plain]{  \heading{Plan de la Présentation}
\ecmsetupplan
\begin{itemize}
  \item[\small $\Diamond$] Introduction
  \item[\small $\Diamond$] Fonctionnement 
  \item[\small $\Diamond$] Applications
  \item[\small $\Diamond$] {\color{purple}\textbf{Technique}}
  \item[\small $\Diamond$] Communauté
  \item[\small $\Diamond$] Conclusions
\end{itemize}
}


\tikzstyle{data} = [draw,rounded corners,fill=blue!20]
\tikzstyle{database} = [draw,cylinder,shape border rotate=90,aspect=0.07,fill=purple!40]
\tikzstyle{tool} = [draw,thin,fill=green!30,text centered,drop shadow]




\frame { \heading{Composants logiciels} \vfill

\begin{tikzpicture}[auto,>=latex',font={\sffamily\tiny}]
    \node[data] (0,0) (GPS) {traces GPS};
    \node[data,right of=GPS,node distance=1.5cm] (yahoo) {imagerie Yahoo!};
    \node[data,right of=yahoo,node distance=1.5cm] (cadastre) {cadastre};
    \node[coordinate,below of=yahoo,node distance=0.5cm] (data) {};
    \node[tool,below of=data,xshift=-0.6cm,node distance=0.5cm] (JOSM) {JOSM};
    \node[tool,below of=data,xshift=0.6cm,node distance=0.5cm] (potlatch) {Potlatch};
    \path[draw] (GPS) -- (data);
    \path[draw] (yahoo) --  (data);
    \path[draw] (cadastre) -- (data);  
    \path[draw,->] (data) -- (JOSM);
    \path[draw,->] (data) -- (potlatch);
    \pause
    
    \node[tool,below of=data,node distance=1.5cm] (API) {API 0.6};
    \path[draw,<->] (JOSM) -- node[yshift=-0.2cm] {REST} (API);
    \path[draw,<->] (potlatch) -- (API);

    \node[database,below of=API] (PG) {backend PostgreSQL};
    \path[draw,<->] (API) -- (PG);
    \pause

    \node[data,left of=API,node distance=1.5cm] (geodata) {données GIS};
    \path[draw,->] (geodata) -- (API);
    \pause
    
    \node[tool,right of=PG,node distance=2cm] (osmosis) {osmosis};
    \path[draw,->] (PG) -- (osmosis);
    \node[database,right of=osmosis,node distance=2cm] (planet) {dumps ``planet''};
    \path[draw,->] (osmosis) -- (planet);
    \pause
    
    \node[tool,above of=planet] (osm2pgsql) {osm2pgsql};
    \path[draw,->] (planet) -- (osm2pgsql);
    \node[database,above of=osm2pgsql] (postgis) {PostGIS};
    \path[draw,->] (osm2pgsql) -- (postgis);
    \node[tool,above of=postgis,text width=1cm,text centered] (mapnik) {mapnik \& mod\_tile};
    \path[draw,->] (postgis) -- (mapnik);
    \node[tool,right of=mapnik,node distance=1.8cm] (style) {feuilles de style};
    \path[draw,->] (style) -- (mapnik);
    \node[database,above of=mapnik,node distance=0.8cm] (tiles) {tuiles};
    \path[draw,->] (mapnik) -- (tiles);
    \node[above of=tiles,node distance=1.1cm] (slippy)
    {\includegraphics[width=0.7cm]{figures/OpenLayers-logo}\hskip-0.8cm\raisebox{17pt}{OpenLayers}};
    \path[draw,->] (tiles) -- (slippy);
    \node[right of=slippy,node distance=2cm] (browser){\includegraphics[width=0.7cm]{figures/firefox-logo}};
    \path[draw,<-] (slippy) -- (browser);
    \pause
    
    \node[tool,right of=planet,node distance=2cm] (XAPI) {XAPI};
    \path[draw,->] (planet) -- (XAPI);
    \node[tool,right of=planet,node distance=2cm,yshift=1cm] (namefinder) {namefinder};
    \path[draw,->] (planet) -- (namefinder);
    \node[database,below of=planet,node distance=0.7cm] (mirror) {planet mirrors};
    \path[draw,->] (planet) -- (mirror);

\end{tikzpicture}
}


%% http://svn.openstreetmap.org/applications/utils/osmosis/trunk/script/contrib/apidb_0.6.sql
%% http://wiki.openstreetmap.org/wiki/Database_schema
%% http://en.wikipedia.org/wiki/Morton_number_(number_theory)
\frame { \heading{Backend PostgreSQL} \vfill

  \begin{itemize}
  \item 1.23 To sur disque (dont 690 Go pour l'historique, 100 Go pour les
  points GPS)
  
  \item Moyenne de 3 millions de tuples lus par seconde

  \item Moyenne de 500 ajouts/modifs/suppressions par seconde

  \item Utilisation de quadtiles  % FIXME expand
  \end{itemize}
}


\frame { \heading{Fonctionnement API} \vfill

Exemples de requêtes:

\begin{itemize}
\item \url{http://api.openstreetmap.org/api/0.6/map?bbox=left,bottom,right,top}
\end{itemize}
}


\frame { \heading{osm2pgsql} \vfill

   \texttt{osm2pgsql -E 900913 -d myDataBase france.osm}

This will process the XML information and load the data into a PostGIS database called
myDataBase. The -E defines the projection of the source data, which in this case is 900913,
the projection used in Google Maps, and the default for OpenStreetMap.

If successful, the database will contain the following tables:

planet\_osm\_line
planet\_osm\_point
planet\_osm\_polygon
planet\_osm\_roads

There are two different tables that contain line data, planet\_osm\_line and planet\_osm\_roads. The
former includes railroads, subways, and other linear information. The latter is made up exclusively
of roads. The planet\_osm\_point table has a range of data: subway stations, shopping centers,
universities, and even brothels. Lastly the planet\_osm\_polygon table has, but is not limited to,
parks, bodies of water, and even buildings in certain urban areas.

}



\begin{frame}[fragile]{Format XML planet}

\tiny
\begin{verbatim}
<?xml version='1.0' encoding='UTF-8'?>
<osm version="0.5" generator="Osmosis 0.26">
  <bound box="41.33878,-5.14222,51.09280,9.56156" origin="http://www.openstreetmap.org/api/0.5"/>
 <node id="475037" timestamp="2008-09-18T07:33:10Z" user="HeikoE" lat="48.4949265" lon="7.7839893"/>
  <node id="475038" timestamp="2008-09-18T07:33:10Z" user="HeikoE" lat="48.4946876" lon="7.7814487"/>
  <node id="477010" timestamp="2008-09-18T07:33:10Z" user="Pieren" lat="48.4607686" lon="7.5112136"/>
  <node id="477015" timestamp="2008-09-18T07:33:10Z" user="Pieren" lat="48.4671357" lon="7.5124533"/>
  ...
 <way id="2788576" timestamp="2008-09-18T07:33:10Z" user="Alban">
    <nd ref="12606591"/>
    <nd ref="14471355"/>
    <nd ref="12606592"/>
    <tag k="name" v="Rue Antoine de Saint Exupery"/>
    <tag k="created_by" v="Editop"/>
    <tag k="highway" v="residential"/>
  </way>
  ...
   <relation id="960" timestamp="2008-09-18T07:33:10Z" user="Drexl">
    <member type="way" ref="8157759" role=""/>
    <member type="way" ref="8157772" role=""/>
    <tag k="type" v="multipolygon"/>
    <tag k="created_by" v="Potlatch 0.9c"/>
  </relation>
\end{verbatim}

\end{frame}



\begin{frame}[fragile]{Requêtes GIS avec PostGIS}

\begin{verbatim}
SELECT
  p.name,
  ST_Distance(ST_Transform(p.way,3348),
  ST_Transform(u.way,3348) )
  FROM planet_osm_point p, planet_osm_point u
  where p.amenity='pub'
  AND u.name='University of Ottawa' AND
  ST_Distance(p.way,u.way)<1;
\end{verbatim}

\end{frame}



\frame{ \heading{Osmose}



}

%% EOF
