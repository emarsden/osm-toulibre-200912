%% osm-conclusions.tex
%%
%% Time-stamp: <2009-06-15 emarsden>


\section{Conclusions}

\subsection{Contribuer}

\frame{ \heading{Comment contribuer?} \vfill

  \begin{itemize}
  \item uploader ses traces GPS
  \item cartographier (JOSM, Potlatch)
%  \item fournir du CPU (\verb|tiles@home|)
  \item documenter (traduction francophone du wiki)
  \item promouvoir le projet (GUL, presse, \ldots{})
  \item participer au développement des logiciels
  \item rapporter des bugs, proposer des correctifs
  \item suggérer des améliorations de l'ergonomie des outils
  \end{itemize}

  OpenStreetMap est un projet libre: toute contribution est importante.
}


\subsection{Perspectives}

\frame{ \heading{Perspectives} \vfill

  \begin{itemize}
  \item rollback en cas d'erreur volontaire ou malveillant
  \item[{\color{darkyellow}\Lightning}] outils pour gérer un «~édit war~»:
    \begin{itemize}
    \item prévention: permettre des rendus par langue
    \item locks par objet/zone géographique contestée 
    \item désignation de modérateurs
    \end{itemize}
  \item processus de revue, alerte email en cas de modif zone reviewé
  \item couche permettant scribble / sketch
  \item outil pour signaler une erreur facilement
  \item éventuel changement de licence: CCBYSA $\rightarrow$ ``Open Database Licence''
  \end{itemize}
}



\frame{ \heading{Conclusions} \vfill

  Intérêts des cartes libres:
  \begin{itemize}
  \item utilisation libre des données cartographiques
    \begin{itemize}
    \item créer un logiciel de navigation libre
    \item créer des cartes spécifiques liés à ses propres intérêts
    \item illustrer des documents libres
    \end{itemize}

  \item permettre de corriger des erreurs dans les cartes (rues
    devenues en sens unique, \ldots{})
  \end{itemize}

  \bigskip
  
  Limites:
  \begin{itemize}
  \item couverture encore très inférieure aux solutions propriétaires
    en France
  \item qualité des données non garantie (vandalisme, \ldots{})
  \end{itemize}

}

\frame{ \heading{Conclusions} \vfill

\hspace{0.5cm}
\begin{beamerboxesrounded}[width=0.99\textwidth,scheme=alert,shadow=true]{}\raggedright\sffamily
  OSM permet de constituer et enrichir des données cartographiques
  \textbf{libres} portant sur des thématiques originales, non couvertes par les
  producteurs institutionnels et privés. 
\end{beamerboxesrounded}
\bigskip

Limites:
\begin{itemize}
\item attention à l'utopisme
\item ne pas penser que les données
géographiques libres produites peuvent se suppléer aux données des
producteurs publics institutionnels
\end{itemize}

Le développement des données libres repose sur:
\begin{itemize}
\item l'utilisation de licences adaptées
\item l'action des autorités publiques à tous les niveaux, pour diffuser
librement et gratuitement les données produites sur fonds publics
\end{itemize}

}

% Convention d'Aarhus
%   Signée le 15 juin 1998 ratifiée en France par la
%   loi du 28 février 2002 et du 26 octobre 2005
% • les trois piliers de la Convention
%    – développer l’accès du public à l’information détenue
%      par les autorités publiques,
%    – favoriser la participation du public à la prise des
%      décisions liées à l’environnement,
%    – étendre les conditions d’accès à la justice.



\frame{ \heading{Liens} \vfill

  \begin{itemize}
  \item Site web: \url{http://openstreetmap.org/}
  \item Courriel: \url{talk-fr@openstreetmap.org}
  \item IRC: \#osm-fr sur le serveur \texttt{irc.oftc.net}
  \item Animations de l'évolution du projet: \url{http://www.jabberworld.org/osm/}
  \end{itemize}

  \vfill \small
  Cette présentation est diffusable selon les termes de la license
  CC-BY-SA 2.0. Des éléments ont été repris de présentations préparées
  par:
  \begin{itemize}
  \item Émmanuel Garette
  \item Frederik Ramm
  \item Sylvain Beorchia \& Thomas Walraet
  \end{itemize}
}


%% EOF
